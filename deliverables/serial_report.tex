\documentclass[11pt]{article}
\usepackage[margin=1in]{geometry}
\usepackage{amsmath}
\usepackage{amssymb}
\usepackage{algorithm}
\usepackage{algpseudocode}
\usepackage{graphicx}
\usepackage{booktabs}
\usepackage{hyperref}

\title{Variational Quantum Eigensolver for H$_2$: \\
Serial Implementation and Parallelization Strategy}
\author{MA453 - High Performance Computing}
\date{Fall 2025}

\begin{document}

\maketitle

\section{Introduction}

The Variational Quantum Eigensolver (VQE) is a hybrid quantum-classical algorithm designed to compute ground state energies of quantum systems. This report describes a serial implementation of VQE for the hydrogen molecule (H$_2$) and outlines a strategy for high-performance computing optimization.

\section{Mathematical Model}

\subsection{The Variational Principle}

VQE exploits the variational principle of quantum mechanics: for any trial wavefunction $|\psi(\theta)\rangle$ parameterized by $\theta$, the expectation value of the Hamiltonian provides an upper bound on the ground state energy:

\begin{equation}
E(\theta) = \langle \psi(\theta) | H | \psi(\theta) \rangle \geq E_0
\end{equation}

where $E_0$ is the true ground state energy and $H$ is the molecular Hamiltonian.

\subsection{Molecular Hamiltonian}

For the H$_2$ molecule, the electronic Hamiltonian in second quantization is:

\begin{equation}
H = \sum_{i,j} h_{ij} a_i^\dagger a_j + \frac{1}{2}\sum_{i,j,k,\ell} h_{ij k\ell} a_i^\dagger a_j^\dagger a_k a_\ell
\end{equation}

where $h_{ij}$ are one-electron integrals and $h_{ij k\ell}$ are two-electron integrals computed using the Hartree-Fock method with the STO-3G basis set. This Hamiltonian is mapped to qubit operators via the Jordan-Wigner transformation.

\subsection{Quantum Circuit Ansatz}

The trial wavefunction is prepared using a parameterized quantum circuit:

\begin{equation}
|\psi(\theta)\rangle = U(\theta) |\text{HF}\rangle
\end{equation}

where $|\text{HF}\rangle$ is the Hartree-Fock reference state and $U(\theta)$ is a unitary operator implemented as a double excitation gate:

\begin{equation}
U(\theta) = \exp\left(-i\frac{\theta}{2}(a_0^\dagger a_1^\dagger a_2 a_3 - a_3^\dagger a_2^\dagger a_1 a_0)\right)
\end{equation}

This ansatz captures the dominant electron correlation effects in H$_2$ while requiring only a single variational parameter.

\subsection{Optimization Problem}

The VQE algorithm solves:

\begin{equation}
\theta^* = \arg\min_\theta E(\theta) = \arg\min_\theta \langle \psi(\theta) | H | \psi(\theta) \rangle
\end{equation}

using the Adam optimizer with learning rate $\alpha = 0.01$ for 200 iterations per bond configuration.

\section{Computational Approach}

\subsection{Problem Structure}

The computational task consists of computing the potential energy surface by evaluating $E(\theta^*)$ for $N_b = 40$ bond lengths in the range $[0.1, 3.0]$ \AA. For each bond length $d_i$:

\begin{enumerate}
\item Generate molecular Hamiltonian $H(d_i)$ using Hartree-Fock
\item Initialize variational parameters $\theta_0 = 0$
\item Optimize: $\theta^*_i = \text{Adam}(E(\theta), \theta_0, N_{\text{iter}} = 200)$
\item Store ground state energy $E_i = E(\theta^*_i)$
\end{enumerate}

\subsection{Serial Algorithm}

\begin{algorithm}
\caption{Serial VQE for H$_2$ Potential Energy Surface}
\begin{algorithmic}[1]
\State \textbf{Input:} Bond lengths $\{d_1, \ldots, d_{40}\}$
\State \textbf{Output:} Energies $\{E_1, \ldots, E_{40}\}$
\State
\For{$i = 1$ to $40$}
    \State Generate $H(d_i)$ using Hartree-Fock
    \State $\theta \gets 0$
    \For{$j = 1$ to $200$}
        \State $E \gets \langle \psi(\theta) | H(d_i) | \psi(\theta) \rangle$ \Comment{Quantum circuit}
        \State $\theta \gets \text{Adam\_step}(\theta, \nabla_\theta E)$
    \EndFor
    \State $E_i \gets E(\theta)$
\EndFor
\State \Return $\{E_1, \ldots, E_{40}\}$
\end{algorithmic}
\end{algorithm}

\subsection{Computational Complexity}

Each quantum circuit evaluation requires $O(4^n)$ operations for an $n$-qubit system using classical simulation. For our 4-qubit system:

\begin{itemize}
\item Circuit evaluations per bond length: 200
\item Total circuit evaluations: $40 \times 200 = 8{,}000$
\item State vector dimension: $2^4 = 16$
\end{itemize}

\section{Performance Baseline}

The serial implementation was benchmarked on a desktop CPU:

\begin{table}[h]
\centering
\begin{tabular}{lr}
\toprule
\textbf{Metric} & \textbf{Value} \\
\midrule
Total Runtime & 50.64 seconds \\
Time per Bond Length & 1.27 seconds \\
Time per VQE Iteration & 6.3 ms \\
Circuit Evaluations/sec & 157.98 \\
\bottomrule
\end{tabular}
\caption{Serial implementation performance metrics.}
\end{table}

\section{Parallelization Strategy}

\subsection{Embarrassingly Parallel Structure}

The outer loop over bond lengths exhibits embarrassing parallelism: each bond length calculation is independent with no data dependencies. Mathematically:

\begin{equation}
E_i = f(d_i) \quad \text{for } i = 1, \ldots, 40
\end{equation}

where $f$ is the VQE optimization procedure. This enables straightforward parallelization.

\subsection{Proposed Optimization Phases}

\textbf{Phase 1: JIT Compilation}
\begin{itemize}
\item Apply JAX just-in-time compilation to cost function
\item Expected speedup: 2--5$\times$ from optimized circuit execution
\end{itemize}

\textbf{Phase 2: Shared-Memory Parallelism}
\begin{itemize}
\item Parallelize outer loop using Python multiprocessing
\item Distribute $N_b = 40$ bond lengths across $p$ cores
\item Expected speedup: $S_p = p$ (ideal), 0.8$p$ (realistic) for $p \leq 8$
\end{itemize}

\textbf{Phase 3: Distributed-Memory Parallelism}
\begin{itemize}
\item Use Ray for task-based distribution across cluster nodes
\item Each node computes subset of bond lengths
\item Expected speedup: Linear up to $p = 40$ nodes
\end{itemize}

\subsection{Performance Model}

For strong scaling with $p$ processors and Amdahl's law:

\begin{equation}
S_p = \frac{1}{f_s + \frac{f_p}{p}}
\end{equation}

where $f_s \approx 0.05$ (setup overhead) and $f_p \approx 0.95$ (parallel fraction). Predicted speedup:

\begin{table}[h]
\centering
\begin{tabular}{lrr}
\toprule
\textbf{Processors} & \textbf{Ideal Speedup} & \textbf{Predicted Speedup} \\
\midrule
4 & 4.0$\times$ & 3.48$\times$ \\
8 & 8.0$\times$ & 6.15$\times$ \\
16 & 16.0$\times$ & 10.39$\times$ \\
40 & 40.0$\times$ & 18.87$\times$ \\
\bottomrule
\end{tabular}
\caption{Predicted parallel speedup using Amdahl's law.}
\end{table}

\subsection{Efficiency Metrics}

We will measure:
\begin{itemize}
\item \textbf{Speedup}: $S_p = T_1 / T_p$ where $T_p$ is runtime with $p$ processors
\item \textbf{Efficiency}: $E_p = S_p / p$
\item \textbf{Strong scaling}: Fixed problem size, varying $p$
\item \textbf{Weak scaling}: Problem size scales with $p$
\end{itemize}

\section{Conclusion}

The serial VQE implementation successfully computes the H$_2$ potential energy surface with well-characterized performance (0.84 minutes for 40 bond lengths). The embarrassingly parallel structure of the outer loop provides clear opportunities for HPC optimization with expected near-linear scaling up to 40 processors. Future work will implement the three-phase parallelization strategy and validate the performance predictions.

\end{document}
